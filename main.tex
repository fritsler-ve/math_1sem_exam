\documentclass{article}
\usepackage[utf8]{inputenc}
\usepackage[T2A] {fontenc}

\title{Математический анализ, I семестр}
\author{fritzlerv}
\date{August 2019}

\usepackage[english, russian]{babel}
\usepackage{graphicx}
\usepackage{amsmath}
\usepackage{amsfonts}
\usepackage{amssymb}

\begin{document}

\maketitle

\newpage

\tableofcontents

\newpage

\section{Последовательности}

\noindent\fbox{%
    \parbox{\textwidth}{%
        \textbf{Последовательность} -- отображение из $\mathbb{N}$ в $\mathbb{R}$
        
        \textbf{Предел последовательности} -- такое число $a$, что  для последовательности $a_n$ выполняется:
        \[ \forall\epsilon>0\exists N_{\epsilon}\in \mathbb{N}: \forall N > N_{\epsilon} \rightarrow |a_N - a| < \epsilon\]
    }%
}

\subsection{Единственность предела сходящейся последовательности}

\textbf{Теорема:} любая сходящаяся последовательность обладает одним и только одним пределом.

\textbf{Доказательство:} От противного. Предположим, что у последовательности $a_n$ одновременно существует два предела, $l_1$ и $l_2$, причем $l_1 > l_2$. 

Определение предела последовательности: 

\[\forall \epsilon > 0 \exists N_0 \in \mathbb{N}: \forall N > N_0 \rightarrow |a_N - l| < \epsilon\]

Отсюда следует, что при $\epsilon < l_1 - l_2$ бесконечное число элементов последовательности заключено в интервале $\left ( l_1 - \epsilon, l_2 + \epsilon \right )$, а при уменьшении $\epsilon$ этот интервал вырождается, следовательно, мы получили противоречие и теорема доказана.

\subsection{Ограниченность сходящейся последовательности}

\textbf{Теорема:} Если последовательность сходится, то она ограничена.

\textbf{Доказательство:} Руководствуясь определением предела последовательности ($\forall \epsilon > 0 \exists N_0 \in \mathbb{N}: \forall N > N_0 \rightarrow |a_N - l| < \epsilon$) получим, что при фиксированном $\epsilon$ в интервале $\left ( l - \epsilon, l + \epsilon  \right )$ содержатся все члены последовательности кроме конечного числа, следовательно, это множество элементов ограничено. Дополнение этого множества также ограничено, но уже в силу того, что число элементов конечно. Что и требовалось доказать.

\subsection{Теорема о переходе к пределу в неравенстве для двух последовательностей}

\textbf{Теорема:} если существуют две сходящихся последовательности $a_n$ и $b_n$, и выполняется следующее: $\exists N_0 \in \mathbb{N}: \forall N > N_0 \rightarrow a_N < b_N$ , то $\lim_{N \rightarrow \infty}a_N < \lim_{N\rightarrow\infty}b_N$

\textbf{Доказательство:} От противного. Пусть $\lim_{N \rightarrow \infty}a_n > \lim_{N \rightarrow \infty}b_n$. Тогда $\forall\epsilon>0 \exists N_0 \in \mathbb{N}: \forall N > N_0 \rightarrow a_N > \lim_{N \rightarrow \infty}a_n - \epsilon, b_N < \lim_{N \rightarrow \infty} + \epsilon$. Если же взять $\epsilon < \frac{\lim_{N \rightarrow \infty}a_n - \lim_{N \rightarrow \infty}b_n}{2}$ получим, что $a_N > b_N$, что, в свою очередь, противоречит условию.

\subsection{Теорема о трёх последовательностях}

\textbf{Теорема:} если существуют три последовательности $a_n$, $b_n$ и $c_n$, и выполняется следующее: $\exists N_0 \in \mathbb{N}: \forall N > N_0 \rightarrow a_N < c_N < b_N$ и $\lim_{N \rightarrow \infty}a_N = \lim_{N \rightarrow \infty}b_N$, то $\exists \lim_{N \rightarrow \infty}c_N$ и $\lim_{N \rightarrow \infty}a_N = \lim_{N \rightarrow \infty}c_N = \lim_{N\rightarrow\infty}b_N$

\textbf{Доказательство:} возьмём последовательности $b2_n$ и $c2_n$, где $b2_n = b_n - a_n$ и $c2_n = c_n - a_n$. Предел последовательности $b2_n$ равен нулю. В силу двойного неравенства из условия получим, что все элементы этих последовательностей неотрицательны. Теперь, руководствуясь тем, что каждый член последоватльности $c2_n$ меньше аналогичного элемента $b2_n$ и определением предела последовательности, получим, что $\lim_{N \rightarrow\infty}c2_N = 0$. Поскольку $c2_n$ получена биективным преобразованием из $c_n$ выведем, что $\lim_{N \rightarrow\infty}c_N = \lim_{N\rightarrow\infty}a_N$. Что и требовалось доказать.

\subsection{Бесконечно малые последовательности и их свойства}

\noindent\fbox{%
    \parbox{\textwidth}{%
        \textbf{Бесконечно малая последовательность} -- последовательность, предел которой равен нулю. 
    }%
}

\textbf{Теорема:} сумма, разность и произведение конечного числа бесконечно малых последовательностей - бесконечно малая последовательнось.

\textbf{Доказательство:} докажем данную теорему для суммы двух последовательностей. Запишем незначительно изменённое определение предела для каждой из этих последовательностей:

\[ \forall \epsilon > 0 \exists N_1 \in \mathbb{N}: \forall N > N_1 \rightarrow |a_1N| < \frac{\epsilon}{2} \]
\[ \forall \epsilon > 0 \exists N_2 \in \mathbb{N}: \forall N > N_2 \rightarrow |a_2N| < \frac{\epsilon}{2} \]

Возьмём $N_3 = \max(N_1, N_2)$, и соединив формулы выше, получим следующее:

\[ \forall \epsilon > 0 \exists N_3 \in \mathbb{N}: \forall N > N_3 \rightarrow |a_1N + a_2N| < \epsilon \]

Утверждение доказано, доказательство для вычитания получается умножением любой из последовательностей на $-1$, для умножения -- поиском во второй последовательности такого $N_0$, что $\forall N > N_0 \rightarrow |a_2N| < 1$.
Доказательство этих свойств для любого конечного числа последовательностей получается поочередным применением вышеуказанного доказательства для уже готовой комбинации и новой неиспользованной последовательности.

\textbf{Утверждение:} произведение бесконечно малой и ограниченной последовательностей является бесконечно малой последовательностью. В доказательстве заменяем это произведение на сумму бесконечно маллых последовательностей, где количество слагаемых -- некоторая верхняя грань ограниченной последовательности.

\textbf{Утверждение:} любую сходящуюся последовательность можно представить в виде суммы бесконечно малой последовательности и последовательности, каждый член которой равен пределу исходной. Доказательство этого факта напрямую вытекает из определения предела последовательности.

\subsection{Арифметические свойства сходящихся последовательностей}

\textbf{Теорема:} пусть есть $a_n$ $b_n$ $\lim_{N\rightarrow\infty}a_N = A$ $\lim_{N\rightarrow\infty} b_N = B$. Тогда:

\begin{itemize}
  \item $\lim_{N\rightarrow\infty}(a_N \pm b_N) = A \pm B$
  \item $\lim_{N\rightarrow\infty}(a_N * b_N) = A * B$
  \item Если $\forall N \rightarrow b_N \neq 0$ и $B \neq 0$ то $\lim_{N\rightarrow\infty}(\frac{a_N}{b_N}) = \frac{A}{B}$
\end{itemize}

\textbf{Доказательство:} докажем последний пункт, поскольку доказательство остальных вытекает из свойств бесконечно малых последовательностей.

Пусть: $a1_n = A - a_n$, $b1_n = B - b_n$. Теперь докажем, что $c_n = \frac{a_n}{b_n} - \frac{A}{B}$ -- бесконечно малая последовательность, что, в свою очередь, будет означать верность исходного утверждения.

\[ \frac{a_n}{b_n} - \frac{A}{B} = \frac{A - a1_n}{B - b1_n} - \frac{A}{B} = \frac{A}{B - b1_n} - \frac{A}{B} - \frac{a1_n}{B - b1_m} = \frac{A * b1_n}{B(B - b1_n)} - \frac{a1_n}{B - b1_n}\]

После этого, используя свойства бесконечно малых последовательностей, нетрудно доказать, что $c_n$ -- бесконечно малая последовательность, из чего, в свою очередь, следует верность исходного утверждения.

\subsection{Теорема Кантора о вложенных отрезках}

\textbf{Теорема:} возьмём последовательность пар чисел $(a_n, b_n)$ такую, что $\forall n \rightarrow a_n < a_{n + 1} < ... < b_{n + 1} < b_n$. Для любой последовательности вложенных отрезков существует хотя бы одна точка $c$, принадлежащая каждому из этих отрезков, а если $\lim_{N \rightarrow \infty}(b_N - a_N) = 0$, то такая точка только одна.

\textbf{Доказательство:} 
\begin{itemize}
    \item Множество правых концов отрезка лежит правее множества левых концов отрезка, поскольку: 
\[\forall n, m \rightarrow a_n < b_m\] 
В силу аксиомы непрерывности получим:
\[\forall n, m \exists c: a_n < c < b_m\]
В частности:
\[\forall n \exists c: a_n < c < b_n\]
    \item Единственность такой точки при устремлении к нулю длин отрезков докажем от противного. Предположим, что $\exists c, c' : c - c' \neq 0; \forall n \rightarrow c, c' \in [a_n, b_n]$
    Взяв в пределе из условия теоремы $\epsilon = \frac{c - c'}{2}$ придем к противоречию. Что и требовалось доказать.
\end{itemize}

\subsection{Теорема Больцано-Вейерштрасса об ограниченных последовательностях}

\textbf{Теорема:} из любой ограниченной последовательности можно выделить сходящуюся подпоследовательность.

\textbf{Доказательство:} начнём рекурсивно разделять множество членов последовательности, и выбирать для следующего шага только ту половину, в которой содержится бесконечное число элементов. В результате этого мы получаем систему стягивающихся отрезков. Согласно теореме Кантора существует точка, принадлежащая каждому из этих отрезков. Она и будет частичным пределом, поскольку в любой её окрестности содержится бесконечное число членов последовательности.

\subsection{Критерий Коши сходимости последовательности}

\textbf{Теорема:} функциональная последовательность сходится.

\textbf{Доказательство:} 
\begin{itemize}
    \item \textbf{Необходимость:} пусть $a_n$ -- сходится, $\lim_{n\rightarrow\infty}a_n = a$. Тогда:
    \begin{multline}
     \forall\epsilon > 0 \exists N\in\mathbb{N}:\forall m, n > N \rightarrow \\ |a_n - a| < \frac{\epsilon}{2}; |a_k - a| < \frac{\epsilon}{2} \rightarrow \\
    |a_n - a| + |a_m - a| <= |a_n - a_m| < \epsilon
    \end{multline}
    
    \item \textbf{Достаточность:} сперва, зафиксировав некоторый $\epsilon$ с помощью определения фундаментальной последовательности, докажем ограниченность исходной последовательности. Поскольку эта последовательность ограниченна, у неё есть как минимум один частичный предел. 
    
    Теперь докажем, что частичный предел у данной последовательности только один. В самом деле, если бы их было несколько, то при некотором $\epsilon$ условие фундаментальности бы нарушалось.

    Таким образом, у нас есть последовательность с единственным частичным пределом, т.е. $\overline{a} = \underline{a}$, а значит, последовательность сходится.   
\end{itemize}

\section{Множества}

\subsection{Теорема о точных верхних и нижних гранях}

\textbf{Теорема:} $\forall X : X\neq \varnothing \exists M$, где $M$ - точная верхняя/нижняя грань множества $X$

\textbf{Доказательство:} возьмём любую верхнюю грань множества $X$, и любой элемент этого же множества. Разделим получившийся отрезок на дае части, из них отбросим ту, в которой нет элементов множества $X$. Продолжим делать так дальше, и получим последовательность стягивающихся отрезков, таких, что один конец каждого отрезка является верхней гранью, а второй - меньше либо равен какоиу-то элементу множества.

Согласно т. Кантора о вложенных отрезках, найдётся точка $M$, которая является верхней гранью, и $\forall \epsilon > 0 \exists x\in X: x > M-\epsilon$. Что и требовалось доказать.

\subsection{Прицип Бореля-Лебега}

\noindent\fbox{
    \parbox{\textwidth}{
        \textbf{Определение.} Система множеств $S = {X}$ покрывает множество $A$, если $\forall a\in A \rightarrow a\in\cup S$ или же $\forall a\in A \exists X\in S: a\in X$
    }
}

\textbf{Теорема:} В любой системе интервалов, покрывающих отрезок $[a, b]$, имеется конечная подсистема, покрывающая этот отрезок.

\textbf{Доказательство.} Пусть $S = {X}$ -- система интервалов X, покрывающих отрезок $[a,b] = I_1$. Предположим, что $I_1$ нельзя покрыть конечным числом интервалов. Тогда будем итеративно делить отрезок $[a_n, b_n]$ на две части, и выбирать для следующего шага ту $[a_{n+1}, b_{n+1}]$, которую нельзя покрыть конечным числом интервалов. Получим систему вложенных отрезков. Согласно теореме Кантора, у нас будет существовать точка, принадлежащая всем этим отрезкам, принадлежащая некоторому интервалу из $S$, а это значит, что начиная с некоторого $n$, интервал, покрывающий точку $c$ также покроет и отрезок $[a_n, b_n]$, что противоречит алгоритму выбора интервалов.

\subsection{Теорема Больцано-Вейерштрасса о существовании предельной точки у ограниченного числового множества}

\noindent\fbox{
    \parbox{\textwidth}{
        \textbf{Предельная точка множества $A$} -- такая точка $c$, что $\forall\epsilon > 0 \exists a \in A: a\in U^{\circ}_{\epsilon}(c)$
    }
}

\textbf{Теорема:} Всякое бесконечное ограниченное числовое множество имеет, по крайней мере, одну предельную точку.

\textbf{Доказательство:} пусть $A$ -- бесконечное ограниченное числовое множество. Из его ограниченности следует, что $\exists [b_1, b_2]$, покрывающий это множество. Начнём строить систему вложенных отрезков по следующему алгоритму:
\begin{enumerate}
    \item Разделим отрезок на две равные части
    \item Выберим из них ту, в которой находится бесконечное количество точек исходного множества
\end{enumerate}

После этого, руководствуясь теоремой Кантора о вложенных отрезках, получим, что существует точка, в любой $\epsilon$-окрестности которой находится бесконечное число элементов множества. Следовательно, найденная нами точка -- предельная.

\section{Функция одной переменной}

\noindent\fbox{
    \parbox{\textwidth}{
        \textbf{Функция} -- отображение из $\mathbb{R}$ в $A: A\subseteq \mathbb{R}$
    }
}

\subsection{Эквивалентность определений предела функции в точке}

\noindent\fbox{
    \parbox{\textwidth}{
        $a$ -- предел в точе $x_0$, если:
        
        \textbf{По Коши:} 
        \begin{equation}
            \forall \epsilon > 0 \exists \delta: \forall x\in U^{\circ}_{\delta}x_0 \rightarrow |a - f(x)| < \epsilon
        \end{equation}
        
        \textbf{По Гейне}
        \begin{equation}
            \forall x_n : \forall n \rightarrow x_n \neq x_0, \lim_{N\rightarrow\infty}x_N = x_0; \lim_{N\rightarrow\infty}f(x_N) = a 
        \end{equation}
    }
}

\textbf{Теорема:} определения по Коши и по Гейне эквивалентны.

\textbf{Доказательство:}

\begin{itemize}
    \item \textbf{Коши $\rightarrow$ Гейне}
    
    Возьмём последовательность $\sigma_n : \lim_{N\rightarrow\infty}\sigma_N = 0$. В каждой $\sigma$-окрестности точки $x_0$ выберем случайную точку. Таким образом, получим сходящуюся к $x_0$ последовательность точек $x_n$. Из определение предела по Коши следует, что последовательность $f(x_0 \pm\sigma)$ при $\sigma\rightarrow 0$ сходится к $a$, а это значит, что $\lim_{n\rightarrow\infty}(f(x_n))=a$.
    
    \item \textbf{Гейне $\rightarrow$ Коши}
    
    Докажем от противного. Предположим, что определение по Гейне верно, а по Коши - нет. Отсюда имеем следующую систему:
    \begin{equation}
        \left\{\begin{matrix}
            \forall x_n : \forall n \rightarrow x_n \neq x_0, \lim_{N\rightarrow\infty}x_N = x_0; \lim_{N\rightarrow\infty}f(x_N) = a 
            \\ 
            \exists \epsilon > 0 \forall \delta: \exists x\in U^{\circ}_{\delta}x_0 \rightarrow |a - f(x)| > \epsilon
        \end{matrix}\right.
    \end{equation}
    Взяв второе утверждение и последовательность $\delta_n: \delta_N=\frac{\delta_0}{2^N}$ получим систему неравенств:
    \begin{equation}
        \left\{\begin{matrix}
            0<|x_n - a|<\frac{\delta}{2^n}
            \\
            |f(x_n)-A|>\epsilon
        \end{matrix}\right.
    \end{equation}
    Из верхнего неравенства получаем, что $a$ -- предел последовательности $x_n$, а из нижнего -- что $A$ не может быть пределом последовательности $f(x_n)$, на основании чего получаем противоречие.
\end{itemize}

\subsection{Критерий Коши предела функции в точке}

\textbf{Теорема:} функция имеет приедел в точке $x_0$, если выолняется условие Коши:
\begin{equation}
    \forall\epsilon>0\exists\delta:\forall x',x" \in U_{\delta}(x_0) \rightarrow |f(x") - f(x')| < \epsilon
\end{equation}

\textbf{Доказательство:}
\begin{enumerate}
    \item \textbf{Необходимость.} Пусть $\exists\lim_{x\rightarrow x_0} f(x) \ a\in\mathbb{R}$.
    
    Тогда $\forall\epsilon>0\exists\delta:\forall x'\in U_{\epsilon}(a) \rightarrow |f(x) - a| < \frac{\epsilon}{2}$
    Отсюда получим: $|f(x') - f(x")| \leq |f(x')-a|+|f(x")-a| = \frac{\epsilon + \epsilon}{2} = \epsilon$.
    Что и требовалось доказать.
    
    \item \textbf{Достаточность.} возьмем случайную последовательность $x_n$, сходящуюся к $x_0$. В силу определения предела последовательности, существует такой номер $N_0$, что $\forall N > N_0 \rightarrow f(x_N)\in U_{\epsilon}(f(x_0))$. Тогда $\forall m, k > N_0 \rightarrow |f(x_m) - f(x_k)| < \epsilon$. Таким образом, последовательность $f(x_n)$сходится, а в силу определения по Гейне и наша функция имеет конечный предел.
    
\end{enumerate}

\subsection{Непрерывность сложной функции}

\noindent\fbox{
    \parbox{\textwidth}{
        \textbf{Право- и левосторонние пределы} -- пределы, в которых при нахождении предела точки берутся из правой или левой полуокрестности т.$x_0$ соответственно.
    
        \textbf{Функция $f(x)$ непрерывна} в каждой точке отрезка $[a, b]$, если правосторонний и левосторонний  пределы в каждой точке этого отрезка равны.
    }
}

\textbf{Теорема:} суперпозиция двух непрерывных функций непрерывна.

\textbf{Доказательство:} не буду доказывать.

\subsection{Теоремы Вейерштрасса для непрерывных на отрезке функций}

\textbf{Теорема:}

\textbf{Доказательство:}

\subsection{Теорема Больцано-Коши о нулях функции}

\textbf{Теорема:}

\textbf{Доказательство:}

\subsection{Теорема об обратной функции}

\textbf{Теорема:}

\textbf{Доказательство:}

\subsection{Теорема о непрерывности обратной функции}

\textbf{Теорема:}

\textbf{Доказательство:}

\section{Производные и дифференциалы}

\subsection{Производная и дифференцируемость функции в точке}

\noindent\fbox{
    \parbox{\textwidth}{
        Если функция $f(x)$ определена в некоторой окрестности точки $x_0$, и существует конечный предел:
        
        \begin{equation}
            f'(x_0) = \lim_{\Delta x \rightarrow 0}\frac{\Delta y}{\Delta x} = \lim_{\Delta x \rightarrow 0}\frac{f(x_0 - \Delta x) - f(x_0)}{\Delta x}
        \end{equation}
        
        То этот предел называется \textbf{производной} данной функции в точке $x_0$
    
        \textbf{Функция дифференцируема} в точке $x_0$, если она определена в некоторой $\epsilon$-окрестности данной точки и её приращение можно выразить так:
        \begin{equation}
            \Delta y = A\Delta x + \Delta x * \epsilon(\Delta x)
        \end{equation}
        Где $A \in \mathbb{R}, \lim_{\Delta x \rightarrow 0}(\epsilon(\Delta x)) = 0$
        
        \textbf{Дифференциал} функции в точке -- произведение $A\Delta x$
    }
}

\textbf{Теорема 1} Функция имеет производнуй в точке $x_0$ тогда и только тогда, когда функция дифференцируема в данной точке.

\textbf{Доказательство.}

\[ \Delta f(x) = A*\Delta x + \epsilon(\Delta x)*\Delta x \]

\[ \frac{\Delta f(x)}{\Delta x} = A + \epsilon(\Delta x) \]

Отсюда следует, что при дифференцируемости функции в данной точке, в ней определена производная, а при переходе к пределу при $\Delta x \rightarrow 0$ из существования в данной точке производной вытекает дифференцируемость. 

\textbf{Теорема 2.} Если $\exists f'(x_0)$, то $f(x)$ непрерывна в некоторой окрестности точки $x_0$.

\textbf{Доказательство.} доделать позже.

\subsection{Дифференцируемость сложной функции в точке}

\textbf{Теорема:}

\textbf{Доказательство:}

\subsection{Теоремы Ферма, Ролля, Лагранжа и Коши}

\textbf{Теорема:}

\textbf{Доказательство:}

\subsection{Правила Лопиталя}

\textbf{Теорема:}

\textbf{Доказательство:}

\subsection{Теорема Тейлора}

\textbf{Теорема:}

\textbf{Доказательство:}

\subsection{Достаточные условия экстремума}

\textbf{Теорема:}

\textbf{Доказательство:}

\subsection{Точки возрастания функции}

\textbf{Теорема:}

\textbf{Доказательство:}

\end{document}
